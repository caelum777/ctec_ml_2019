\documentclass[]{article}
\usepackage{lmodern}
\usepackage{amssymb,amsmath}
\usepackage{ifxetex,ifluatex}
\usepackage{fixltx2e} % provides \textsubscript
\ifnum 0\ifxetex 1\fi\ifluatex 1\fi=0 % if pdftex
  \usepackage[T1]{fontenc}
  \usepackage[utf8]{inputenc}
\else % if luatex or xelatex
  \ifxetex
    \usepackage{mathspec}
  \else
    \usepackage{fontspec}
  \fi
  \defaultfontfeatures{Ligatures=TeX,Scale=MatchLowercase}
\fi
% use upquote if available, for straight quotes in verbatim environments
\IfFileExists{upquote.sty}{\usepackage{upquote}}{}
% use microtype if available
\IfFileExists{microtype.sty}{%
\usepackage{microtype}
\UseMicrotypeSet[protrusion]{basicmath} % disable protrusion for tt fonts
}{}
\usepackage[margin=1in]{geometry}
\usepackage{hyperref}
\hypersetup{unicode=true,
            pdftitle={Regresion},
            pdfborder={0 0 0},
            breaklinks=true}
\urlstyle{same}  % don't use monospace font for urls
\usepackage{color}
\usepackage{fancyvrb}
\newcommand{\VerbBar}{|}
\newcommand{\VERB}{\Verb[commandchars=\\\{\}]}
\DefineVerbatimEnvironment{Highlighting}{Verbatim}{commandchars=\\\{\}}
% Add ',fontsize=\small' for more characters per line
\usepackage{framed}
\definecolor{shadecolor}{RGB}{248,248,248}
\newenvironment{Shaded}{\begin{snugshade}}{\end{snugshade}}
\newcommand{\AlertTok}[1]{\textcolor[rgb]{0.94,0.16,0.16}{#1}}
\newcommand{\AnnotationTok}[1]{\textcolor[rgb]{0.56,0.35,0.01}{\textbf{\textit{#1}}}}
\newcommand{\AttributeTok}[1]{\textcolor[rgb]{0.77,0.63,0.00}{#1}}
\newcommand{\BaseNTok}[1]{\textcolor[rgb]{0.00,0.00,0.81}{#1}}
\newcommand{\BuiltInTok}[1]{#1}
\newcommand{\CharTok}[1]{\textcolor[rgb]{0.31,0.60,0.02}{#1}}
\newcommand{\CommentTok}[1]{\textcolor[rgb]{0.56,0.35,0.01}{\textit{#1}}}
\newcommand{\CommentVarTok}[1]{\textcolor[rgb]{0.56,0.35,0.01}{\textbf{\textit{#1}}}}
\newcommand{\ConstantTok}[1]{\textcolor[rgb]{0.00,0.00,0.00}{#1}}
\newcommand{\ControlFlowTok}[1]{\textcolor[rgb]{0.13,0.29,0.53}{\textbf{#1}}}
\newcommand{\DataTypeTok}[1]{\textcolor[rgb]{0.13,0.29,0.53}{#1}}
\newcommand{\DecValTok}[1]{\textcolor[rgb]{0.00,0.00,0.81}{#1}}
\newcommand{\DocumentationTok}[1]{\textcolor[rgb]{0.56,0.35,0.01}{\textbf{\textit{#1}}}}
\newcommand{\ErrorTok}[1]{\textcolor[rgb]{0.64,0.00,0.00}{\textbf{#1}}}
\newcommand{\ExtensionTok}[1]{#1}
\newcommand{\FloatTok}[1]{\textcolor[rgb]{0.00,0.00,0.81}{#1}}
\newcommand{\FunctionTok}[1]{\textcolor[rgb]{0.00,0.00,0.00}{#1}}
\newcommand{\ImportTok}[1]{#1}
\newcommand{\InformationTok}[1]{\textcolor[rgb]{0.56,0.35,0.01}{\textbf{\textit{#1}}}}
\newcommand{\KeywordTok}[1]{\textcolor[rgb]{0.13,0.29,0.53}{\textbf{#1}}}
\newcommand{\NormalTok}[1]{#1}
\newcommand{\OperatorTok}[1]{\textcolor[rgb]{0.81,0.36,0.00}{\textbf{#1}}}
\newcommand{\OtherTok}[1]{\textcolor[rgb]{0.56,0.35,0.01}{#1}}
\newcommand{\PreprocessorTok}[1]{\textcolor[rgb]{0.56,0.35,0.01}{\textit{#1}}}
\newcommand{\RegionMarkerTok}[1]{#1}
\newcommand{\SpecialCharTok}[1]{\textcolor[rgb]{0.00,0.00,0.00}{#1}}
\newcommand{\SpecialStringTok}[1]{\textcolor[rgb]{0.31,0.60,0.02}{#1}}
\newcommand{\StringTok}[1]{\textcolor[rgb]{0.31,0.60,0.02}{#1}}
\newcommand{\VariableTok}[1]{\textcolor[rgb]{0.00,0.00,0.00}{#1}}
\newcommand{\VerbatimStringTok}[1]{\textcolor[rgb]{0.31,0.60,0.02}{#1}}
\newcommand{\WarningTok}[1]{\textcolor[rgb]{0.56,0.35,0.01}{\textbf{\textit{#1}}}}
\usepackage{graphicx,grffile}
\makeatletter
\def\maxwidth{\ifdim\Gin@nat@width>\linewidth\linewidth\else\Gin@nat@width\fi}
\def\maxheight{\ifdim\Gin@nat@height>\textheight\textheight\else\Gin@nat@height\fi}
\makeatother
% Scale images if necessary, so that they will not overflow the page
% margins by default, and it is still possible to overwrite the defaults
% using explicit options in \includegraphics[width, height, ...]{}
\setkeys{Gin}{width=\maxwidth,height=\maxheight,keepaspectratio}
\IfFileExists{parskip.sty}{%
\usepackage{parskip}
}{% else
\setlength{\parindent}{0pt}
\setlength{\parskip}{6pt plus 2pt minus 1pt}
}
\setlength{\emergencystretch}{3em}  % prevent overfull lines
\providecommand{\tightlist}{%
  \setlength{\itemsep}{0pt}\setlength{\parskip}{0pt}}
\setcounter{secnumdepth}{0}
% Redefines (sub)paragraphs to behave more like sections
\ifx\paragraph\undefined\else
\let\oldparagraph\paragraph
\renewcommand{\paragraph}[1]{\oldparagraph{#1}\mbox{}}
\fi
\ifx\subparagraph\undefined\else
\let\oldsubparagraph\subparagraph
\renewcommand{\subparagraph}[1]{\oldsubparagraph{#1}\mbox{}}
\fi

%%% Use protect on footnotes to avoid problems with footnotes in titles
\let\rmarkdownfootnote\footnote%
\def\footnote{\protect\rmarkdownfootnote}

%%% Change title format to be more compact
\usepackage{titling}

% Create subtitle command for use in maketitle
\providecommand{\subtitle}[1]{
  \posttitle{
    \begin{center}\large#1\end{center}
    }
}

\setlength{\droptitle}{-2em}

  \title{Regresion}
    \pretitle{\vspace{\droptitle}\centering\huge}
  \posttitle{\par}
    \author{}
    \preauthor{}\postauthor{}
    \date{}
    \predate{}\postdate{}
  

\begin{document}
\maketitle

\begin{Shaded}
\begin{Highlighting}[]
\KeywordTok{library}\NormalTok{(GGally)}
\end{Highlighting}
\end{Shaded}

\begin{verbatim}
## Warning: package 'GGally' was built under R version 3.6.1
\end{verbatim}

\begin{verbatim}
## Loading required package: ggplot2
\end{verbatim}

\begin{verbatim}
## Registered S3 method overwritten by 'GGally':
##   method from   
##   +.gg   ggplot2
\end{verbatim}

\begin{Shaded}
\begin{Highlighting}[]
\KeywordTok{library}\NormalTok{(ggplot2)}
\KeywordTok{library}\NormalTok{(caTools)}
\end{Highlighting}
\end{Shaded}

\begin{verbatim}
## Warning: package 'caTools' was built under R version 3.6.1
\end{verbatim}

\begin{Shaded}
\begin{Highlighting}[]
\KeywordTok{library}\NormalTok{(bestNormalize)}
\end{Highlighting}
\end{Shaded}

\begin{verbatim}
## Warning: package 'bestNormalize' was built under R version 3.6.1
\end{verbatim}

\begin{Shaded}
\begin{Highlighting}[]
\KeywordTok{library}\NormalTok{(MASS)}
\end{Highlighting}
\end{Shaded}

\begin{verbatim}
## 
## Attaching package: 'MASS'
\end{verbatim}

\begin{verbatim}
## The following object is masked from 'package:bestNormalize':
## 
##     boxcox
\end{verbatim}

\begin{Shaded}
\begin{Highlighting}[]
\KeywordTok{library}\NormalTok{(lessR)}
\end{Highlighting}
\end{Shaded}

\begin{verbatim}
## Warning: package 'lessR' was built under R version 3.6.1
\end{verbatim}

\begin{verbatim}
## 
## lessR 3.8.8     feedback: gerbing@pdx.edu     web: lessRstats.com/new
## ---------------------------------------------------------------------
## 1. d <- Read("")           Read text, Excel, SPSS, SAS or R data file
##                            d: default data frame, no need for data=
## 2. l <- Read("", var_labels=TRUE)   Read variable labels into l,
##                            required name for data frame of labels
## 3. Help()                  Get help, and, e.g., Help(Read)
## 4. hs(), bc(), or ca()     All histograms, all bar charts, or both
## 5. Plot(X) or Plot(X,Y)    For continuous and categorical variables
## 6. by1= , by2=             Trellis graphics, a plot for each by1, by2
## 7. reg(Y ~ X, Rmd="eg")    Regression with full interpretative output
## 8. style("gray")           Grayscale theme, + many others available
##    style(show=TRUE)        all color/style options and current values
## 9. getColors()             create many styles of color palettes
## 
## lessR parameter names now use _'s. Names with a period are deprecated.
## Ex:  bin_width  instead of  bin.width
\end{verbatim}

\begin{Shaded}
\begin{Highlighting}[]
\KeywordTok{library}\NormalTok{(Metrics)}
\end{Highlighting}
\end{Shaded}

\hypertarget{tarea-3.}{%
\section{Tarea 3.}\label{tarea-3.}}

\hypertarget{regresion-lineal}{%
\section{Regresión lineal}\label{regresion-lineal}}

Análisis del Problema

El desempeño de un automóvil se puede medir de diferentes formas.
Algunas comunes son la cantidad de caballos de fuerza y el rendimiento
del mismo, que se puede resumir en cuantas millas puede recorrer el
automóvil por cada galón de combustible que consume. Para los clientes,
potenciales compradores de un automóvil, este rendimiento es importante
pues puede ayudar a tomar una decisión con respecto a cuál automóvil
comprar (si, por ejemplo, el cliente quiere un auto que rinda por muchas
millas y pueda economizar en la compra de combustible).

Desde este punto de vista, tanto a clientes como a fabricadores de
automóviles, les conviene entender cuál es la relación entre diferentes
características del automóvil y su rendimiento, pues el conocer estas
relaciones les puede ayudar a inferir cuál va a ser la eficiencia del
vehículo a partir de ver los valores de otras características. Para
fabricantes, puede ser importante conocer estas relaciones para saber
cómo hacer cada modelo más eficiente con respecto al anterior.

Entendimiento de los Datos

Con el fin de analizar y tratar de estimar las millas por galón de
diferentes modelos de automóviles, se trabajó con un conjunto de datos
que contiene 398 observaciones y 9 variables:

\begin{itemize}
\tightlist
\item
  mpg (millas por galón): numérica, con un rango de 9 a 46.60.
\item
  cyl (cilindraje): categórica ordinal, con valores posibles de 3, 4, 5,
  6 y 8.
\item
  disp (desplazamiento): numérica, con un rango de 68 a 455.
\item
  hp (caballos de fuerza): numérica, con un rango de 46 a 230 y 6
  valores faltantes.
\item
  weight (peso): numérica, con un rango de 1613 a 5140.
\item
  acc (aceleración): numérica, con un rango de 8 a 24.80.
\item
  model year (año): categórica, con 13 valores diferentes representando
  el año del automóvil.
\item
  origin (origen): categórica, 3 valores posibles: 1, 2, 3.
\item
  model name (nombre del modelo): categórica, con 305 posibles valores.
\end{itemize}

\hypertarget{ejercicios}{%
\section{Ejercicios}\label{ejercicios}}

\begin{enumerate}
\def\labelenumi{\arabic{enumi}.}
\tightlist
\item
  Cargue el archivo auto-mpg\_g.csv en una variable
\end{enumerate}

\begin{Shaded}
\begin{Highlighting}[]
\NormalTok{autos_mpg <-}\StringTok{ }\KeywordTok{read.csv}\NormalTok{(}\StringTok{'auto-mpg_g.csv'}\NormalTok{, }\DataTypeTok{header =}\NormalTok{ T, }\DataTypeTok{na.strings =} \StringTok{'?'}\NormalTok{)}
\KeywordTok{head}\NormalTok{(autos_mpg)}
\end{Highlighting}
\end{Shaded}

\begin{verbatim}
##   mpg cyl disp  hp weight acc model.year origin                 model.name
## 1  18   8  307 130   3504  12         70      1  chevrolet chevelle malibu
## 2  15   8  350 165   3693  12         70      1          buick skylark 320
## 3  18   8  318 150   3436  11         70      1         plymouth satellite
## 4  16   8  304 150   3433  12         70      1              amc rebel sst
## 5  17   8  302 140   3449  11         70      1                ford torino
## 6  15   8  429 198   4341  10         70      1           ford galaxie 500
\end{verbatim}

\begin{Shaded}
\begin{Highlighting}[]
\KeywordTok{set.seed}\NormalTok{(}\DecValTok{3}\NormalTok{)}
\end{Highlighting}
\end{Shaded}

\begin{enumerate}
\def\labelenumi{\arabic{enumi}.}
\setcounter{enumi}{1}
\tightlist
\item
  Utilizando Ggpairs cree un gráfico de los atributos del dataset,
  observe las correlaciones entre atributos
\end{enumerate}

Aplicamos un ggpairs para ver cuales variables tienen más correlacion
con MPG

\begin{Shaded}
\begin{Highlighting}[]
\KeywordTok{ggpairs}\NormalTok{(autos_mpg[}\OperatorTok{-}\KeywordTok{c}\NormalTok{(}\DecValTok{9}\NormalTok{)])}
\end{Highlighting}
\end{Shaded}

\includegraphics{Regresion_files/figure-latex/unnamed-chunk-3-1.pdf}

Luego de analizar la correlaciones se determina que las variables disp y
weight son las más significativas

\begin{enumerate}
\def\labelenumi{\arabic{enumi}.}
\setcounter{enumi}{2}
\tightlist
\item
  Separe los datos en 2 conjuntos, uno de entrenamiento y otro de
  pruebas. Normalmente se trabaja utilizando un 70-80\% de los datos
  para entrenamiento y el resto para pruebas.
\end{enumerate}

Recuerde fijar una semilla para que el documento sea reproducible.

Pista:
\url{https://www.rdocumentation.org/packages/caTools/versions/1.17.1/topics/sample.split}

\begin{Shaded}
\begin{Highlighting}[]
\KeywordTok{set.seed}\NormalTok{(}\DecValTok{3}\NormalTok{)}

\NormalTok{result <-}\StringTok{ }\KeywordTok{sample.split}\NormalTok{(autos_mpg}\OperatorTok{$}\NormalTok{mpg, }\DataTypeTok{SplitRatio=}\FloatTok{0.75}\NormalTok{, }\DataTypeTok{group=}\OtherTok{NULL}\NormalTok{)}

\NormalTok{df_train <-}\StringTok{ }\NormalTok{autos_mpg[result}\OperatorTok{==}\OtherTok{TRUE}\NormalTok{, ]}

\NormalTok{df_test <-}\StringTok{ }\NormalTok{autos_mpg[result}\OperatorTok{==}\OtherTok{FALSE}\NormalTok{, ]}
\end{Highlighting}
\end{Shaded}

\begin{enumerate}
\def\labelenumi{\arabic{enumi}.}
\setcounter{enumi}{3}
\tightlist
\item
  Cree un modelo de regresion lineal utilizando el atributo mpg como la
  variable objetivo y en base a las correlaciones observadas en el
  gráfico del punto 2 escoja al menos dos atributos para usarlos como
  variables predictoras para el modelo.
\end{enumerate}

Pista:
\url{https://www.rdocumentation.org/packages/lessR/versions/1.9.8/topics/reg}

Nota: Al crear el modelo utilice el conjunto de datos de entrenamiento
definido en el punto 3.

Aplicamos un log a las variables predictoras y comparamos los
histogramas antes y despues de aplicar el log. Luego de aplicar varias
pruebas, se determinó que aplicar log a la variable MPG empeora el MSE,
por lo tanto se decide aplicar log solo a las variables predictoras.

\begin{Shaded}
\begin{Highlighting}[]
\KeywordTok{hist}\NormalTok{(df_train}\OperatorTok{$}\NormalTok{disp)}
\end{Highlighting}
\end{Shaded}

\includegraphics{Regresion_files/figure-latex/unnamed-chunk-5-1.pdf}

\begin{Shaded}
\begin{Highlighting}[]
\NormalTok{df_train}\OperatorTok{$}\NormalTok{disp <-}\StringTok{ }\KeywordTok{log}\NormalTok{(df_train}\OperatorTok{$}\NormalTok{disp)}
\KeywordTok{hist}\NormalTok{(df_train}\OperatorTok{$}\NormalTok{disp)}
\end{Highlighting}
\end{Shaded}

\includegraphics{Regresion_files/figure-latex/unnamed-chunk-5-2.pdf}

\begin{Shaded}
\begin{Highlighting}[]
\KeywordTok{hist}\NormalTok{(df_train}\OperatorTok{$}\NormalTok{weight)}
\end{Highlighting}
\end{Shaded}

\includegraphics{Regresion_files/figure-latex/unnamed-chunk-5-3.pdf}

\begin{Shaded}
\begin{Highlighting}[]
\NormalTok{df_train}\OperatorTok{$}\NormalTok{weight <-}\StringTok{ }\KeywordTok{log}\NormalTok{(df_train}\OperatorTok{$}\NormalTok{weight)}
\KeywordTok{hist}\NormalTok{(df_train}\OperatorTok{$}\NormalTok{weight)}
\end{Highlighting}
\end{Shaded}

\includegraphics{Regresion_files/figure-latex/unnamed-chunk-5-4.pdf}

\begin{Shaded}
\begin{Highlighting}[]
\NormalTok{df_test}\OperatorTok{$}\NormalTok{disp <-}\StringTok{ }\KeywordTok{log}\NormalTok{(df_test}\OperatorTok{$}\NormalTok{disp)}
\NormalTok{df_test}\OperatorTok{$}\NormalTok{weight <-}\StringTok{ }\KeywordTok{log}\NormalTok{(df_test}\OperatorTok{$}\NormalTok{weight)}
\end{Highlighting}
\end{Shaded}

\begin{Shaded}
\begin{Highlighting}[]
\NormalTok{reg <-}\StringTok{ }\KeywordTok{lm}\NormalTok{(}\DataTypeTok{formula =}\NormalTok{ mpg }\OperatorTok{~}\StringTok{ }\NormalTok{disp }\OperatorTok{+}\StringTok{ }\NormalTok{weight, }\DataTypeTok{data =}\NormalTok{ df_train)}
\end{Highlighting}
\end{Shaded}

\begin{enumerate}
\def\labelenumi{\arabic{enumi}.}
\setcounter{enumi}{4}
\tightlist
\item
  Realice predicciones utilizando el conjunto de pruebas y evalue el
  resultado con la métrica MSE.
\end{enumerate}

Pista:
\url{https://www.rdocumentation.org/packages/mltools/versions/0.3.5/topics/mse}

\begin{Shaded}
\begin{Highlighting}[]
\NormalTok{preds <-}\StringTok{ }\KeywordTok{predict}\NormalTok{(reg, df_test)}
\NormalTok{Metrics}\OperatorTok{::}\KeywordTok{mse}\NormalTok{(df_test}\OperatorTok{$}\NormalTok{mpg, preds)}
\end{Highlighting}
\end{Shaded}

\begin{verbatim}
## [1] 17.99125
\end{verbatim}

\begin{enumerate}
\def\labelenumi{\arabic{enumi}.}
\setcounter{enumi}{5}
\tightlist
\item
  Opcional
\end{enumerate}

6.a Pruebe varios modelos que utilicen diferentes variables y comparar
los resultados obtenidos

6.b Investigar como implementar en R las técnicas de preprocesado y
normalización vistas en clase y aplicarlas a los datos antes de pasarlos
al modelo.


\end{document}
